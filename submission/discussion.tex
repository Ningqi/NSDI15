\vspace{-.5em}
\Section{Discussion and Future Work}
\label{sec:discussion}

\subsection{Network orchestration and view updates}

Discuss the semantics of view updates: An ideal case of view update
keeps a view's complement constant ... that is, a view update does not
impose side affects that may corrupt other views.

View updates ... briefly introduce existing works, connection to
recent provenance work.
% \Paragraph{Forwarding plane updates via query optimization}
% \Paragraph{Network synthesis via database provenance}

\subsection{Consistent forwarding plane and transaction processing}

\todo{need re-write: but all raw texts are here}

Next, \Sys's \TR provides sequential and recoverable behavior of
concurrent user operations, with \textit{ACID} semantics:
sequentiality ensures user operations proceed \textit{atomically} and
are \textit{isolated} from one another, always leaving the network in
a \textit{consistent} state; and recoverability assures operation
failure does not pollute network state, leaving effects of committed
operation \textit{durable}. Unlike conventional database systems
(DBSes) where the operations, transaction, and their connections are
obvious~\cite{Bernstein:concurrency-recovery,concurrency-recovery-alg,principles-tp,concurrency-ddb},
the interpretation in networking is obscure, as observed in early
works~\cite{hotsdn-transactional-networking,On-Consistent-Updates,of-cpp}.
The inherent dilemma is that transaction, the logical unit that
preserves ACID, usually is meaningful only for network-wide actions;
whereas the enabling mechanism usually is only efficiently enforceable
for switch-level operation. \TTR solves this by leveraging the view
abstraction introduced in \TI: transactions, like other high-level
operations, are specified over views; whereas the enforcing mechanisms
are built on base tables.  For example, if a transaction over a
particular view is processed via two-phase locking, according to that
view (the SQL query), \Sys translates locks over the abstract view
into a set of locks that can be performed locally at the relevant
distributed base tables.


Transactional networking offers an efficient execution abstraction of
user programs in a shared distributed network, freeing the user from
the challenging concurrency and recovery control
problem~\cite{consistency-lock,concurrency-recovery-alg,principles-tp,concurrency-ddb,Tc-ddb,crdb}.

\Paragraph{Transaction preserves ACID properties.} In \Sys, a
transaction is a logical unit of operations that are atomically and
isolated from one another, preserving network consistency and
preventing failure from polluting effects of already committed
transaction. The operations in a transaction are partially-ordered,
defined in the user program. An operation is either a read or write: a
write maps to network (re)configuration in the form of an insert or
delete\footnote{An update is a deletion followed by an insertion} of
records in a base table or view; a read, on the other hand, maps to
packet processing, since packet processing is the effect of ``read''
policy data. An example transaction is the collection of flow events
interleaved with network re-configurations issued by a user
program. By DB concurrency and recovery control, \Sys executes
transactions concurrency while retaining the ACID semantics.

\Paragraph{Transactions on views.}  
Like users interact with \Sys via views, they also conceive
transactions on views. Specifically, we write $(T,v, \overline{op})$
for a transaction $T$ with operation set $\overline{op}=op_1,\cdots
op_n$ on a view $v$.  Users tell \Sys of $T$ by wrapping his program
like the following:
\begin{sql}
Start; program (op1;...opn;) Commit;
\end{sql}
The key to efficient transaction processing for parallel programs is a
scheduler that coordinates data access of operations while preserving
ACID. It has been shown that the scheduling problem decompose to two
sub-problems: resolving conflicting read-write operations, and that
for write-write operations~\cite{Bernstein:concurrency-recovery}.
Read-write conflict occurs between a configuration update (write) and
processing of in-fly traffic (read) that will be affected by the
update; write-write conflict occurs when the updated data items
overlap. A standard scheduler that prevents both conflicts is
two-phase locking~\cite{Bernstein:concurrency-recovery} where a
transaction $(T,v,\overline{op})$ becomes $(T,v, (lock(v),
\overline{op}, unlock(v)))$, that is:
\begin{sql}
Start; lock(v); program (op1;...opn;) unlock(v); Commit;
\end{sql}

\Paragraph{Transactions on base tables at switches.} Transactions on
views are inefficient: a view is by nature application-specific,
typically network-wide, involving multiple distributed nodes (\eg that
forms a path).  Transactions over views require synchronization among
the participant nodes, making \nd{lock(v)} and \nd{unlock(v)} complex
tasks that lack performance.  Hence rather than adding concurrency and
recovery enforcement to views, \Sys implemented them at base tables,
where the locks can be performed locally at individual node.  To
enable this, \Sys translates a transaction $(T,v_{routing},
(lock(v_{routing}),op,unlock(v_{routing})))$ on a path defined by a
routing policy view $v_{routing}$ to a set of base table transactions
$(T,b_i, (lock_{b_i}, op_{b_i}, unlock_{b_i})) $ where $b_i$, the base
table derived from the view
($v_{routing}$). 
Operations on $b_i$ proceed independent of each other. When multiple
$T_i$ is executed, two-phase locking over views is achieved by
switch-level locking that enforce a consensus of partial ordering
among conflicting operations.

% \Paragraph{Relational vs. non-relational database} ...  Map-reduce
% ...  Networking services, unlike the ad-hoc lightweight ... However,
% relational database is a general purpose tool not designed for
% highly-connected data. That is, despite decades of relational
% database query optimization and indexing research, which brought its
% commerical success, those non-relational, graph database for
% example, outperforms relational database on query involving graph
% operations.


% \Paragraph{Non-relational database} ... semi-structured, graph
% database, ... 
