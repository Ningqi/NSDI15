\section{Introduction}
\label{sec:intro}

% To bring about the needed abstraction capable of reconciling what a
% human mind perfects and how today's network performs, we

In the 1980s, operating system (OS) and programming language (PL)
techniques were proven to fall short for mediating between
increasingly complex online data management needs and multiple users
of little computer specialty. Likewise, network operations and
infrastructure development today remain manual and difficult, despite
the recent efforts to separate logical abstractions from physical
infrastructure via OS and PL
means~\cite{ethane-sigcomm07,rethinking-enterprise,shenker-tue}. For
example, SDN introduces network OSes~\cite{onix,nox} that maintain
what is effectively a
% network-wide distributed data structure---switches' FIBs---which are
% manipulated at a logically
network-wide data structure, \eg Network Information Base
(NIB)~\cite{onix}, which is manipulated at a logically centralized
controller through programming
APIs~\cite{composing,sdn-lang-frenetic}. However, the network
data-plane is still distributed and shared, enforcing the operator to
explicitly deal with data redundancy, isolation, consistency, and
recovery by composing complete programs out of wrapping APIs, a task
that is notoriously challenging and tedious.  

% Virtualized networks, on the other hand, enable multiple logical
% networks on a shared infrastructure: the logically simpler
% ``one-big-switch'', though gentler to the tenant using this
% abstraction, requires some effort when it comes to the mapping into
% physical infrastructure; a fuller virtualized abstraction that
% supports arbitrary topology, on the other end of the spectrum, can
% be realized with a more straightforward mapping, but requires
% significant work on the tenant side, actually closer to that needed
% for a physical network\footnote{For example, to isolate user
% programs by different tenants, a fuller virtualized network could
% utilize the tenant network namespace as indexes; whereas the the
% one-big-switch abstraction needs to explicitly inspect user program
% context.}.

% For example, software-defined networks (SDN), by exploiting the idea
% of network operating system, exposes and maintains a separate
% network-wide abstraction -- a network representation (analogous to
% FIB) and control primitive (routing, access control \etc).  From a
% user perspective, the abstraction forms a programming
% API~\cite{composing,sdn-lang-frenetic} -- high-level data-structures
% and operations that manipulate them.  Similarly, virtualized
% networks (VN) separates logical services / functionality from the
% underlying physical networks, gives tenants an abstract network
% representation (virtual topology, address \etc), through APIs over a
% spectrum of granularity.

Similar to the transition of online commercial data management in the
1980s, this paper champions a shift from OS/PL techniques in
networking to database (DB)
techniques~\cite{Abiteboul:1995:alice,db-concept,db-meta}, adapting
the two DB pillars---data independence and transaction processing---to
the domain of networking. Data independence refers to abstractions
that simplify human interaction by hiding data storage and
representation, while transaction processing achieves concurrency and
recovery control without severely hurting performance.  In networking,
\TI offers a mechanism to create high-level abstractions of the
distributed network data-plane, while \TR creates an illusion of
isolated user operations in a shared infrastructure.  By porting data
independence and transaction processing into networking, we propose a
\textbf{D}atabase \textbf{B}ased \textbf{N}etworking (\Sys) design.

% This is analogous to pre\hyp{}database day computerized management
% of commercial data, when programmers write applications over
% data\footnote{Unlike the file-system used in conventional operating
% system, today's SDN OS employs a graph databases which is
% essentially a key\hyp{}value storage optimized for graphs.}
% supported by an operating system: For example, \todo the OS provides
% APIs that enable the applications to read and write the appropriate
% data.

% However, network operation and innovation remain manual and
% difficult, due to the grand challenge in finding the right
% abstractions that strikes a balance between two conflicting goals:
% human convenience and reasonable (feature-rich) system performance,
% over a distributed, shared, unreliable network of heterogeneous
% devices. Existing approaches either achieves simplicity,
% performance, or moderate gain on both both.  On one extreme, L2 LAN
% abstraction, though simple and configuration-free, does not scale
% beyond certain topology (\eg spanning tree) and network size (due to
% flat address); On the other, L3 IP-routing offers an end-to-end
% abstraction that scales to the entire Interest, however, imposes
% administration burden through manual configurations. Abstraction
% sitting in the middle includes SDN and virtual networks.

More specifically, \TI offers a programmable user-level abstraction
exposing a human interface that simplifies user logic, together with a
language that creates the abstraction on demand, and a mechanism that
pushes abstract operation back into the network device.  The key is
that, rather than seeking one kind of pre-defined abstraction that
fits every existing user application\footnote{We use user program to
  refer to any control program, \eg network-wide controller program by
  an administrator, or application programs by end users with limited
  network access.}  (and potentially future
ones)~\cite{sdn-lang-frenetic,composing}, we turn the network into a
distributed relational database, where the base tables are switches'
FIBs, and where we use SQL language to create abstractions called
\emph{(network) views} on demand -- permitting high-level
application-specific perspectives.  For example, end-to-end
reachability policy is nothing but a view -- the abstract network-wide
data-structure selected from the distributed FIBs and restructured in
a form, \eg a relation named \nd{e2e\_policy} of three attributes
\nd{\{flow\_id,source,dest\}}, reducing high-level policy manipulation
to simple DB operations. Similarly, routing policy is a view of
\nd{\{flow\_id, path\_vector\}}. To enable network management directly
on views, \Sys utilizes DB view mechanisms, namely view maintenance
and update. View maintenance keeps abstractions updated as the network
changes (i.e., FIB changes), enabling ad-hoc network verification by
running queries over views against the latest network state. For
example, to verify reachable switches for a flow with id \nd{3} via a
node \nd{W} (waypoint), operator issues the following query:
\begin{sql}
SELECT e.dest
FROM   e2e_policy e NATURAL JOIN routing_policy r
WHERE  e.flow_id = 3 AND (`W' in r.path_vector);
\end{sql}
which returns all the reachable nodes in real time. Conversely, view
update compiles an abstract view operation into a collection of
network FIBs operations, synthesizing the actual FIB implementation
based on higher-level policy constraints. For example, to set up a new
path for flow \nd{4} between node \nd{A} and \nd{B}, the operator
simply inserts this constraint into the policy view:
\begin{sql}
INSERT INTO e2e_policy VALUES (1, 'A', 'B');
\end{sql}
which is translated by \Sys into a set of FIB inserts.
% (Section~\ref{sec:details}).

% switch\footnote{\Sys postulates that future switches shall support
% transaction primitives such as locking, either built
% in hardware, or simulated via software.}

% For example, if transaction processing is implemented by two-phase
% locking, rather than grabbing and releasing global locks over the
% views, which is in-efficient and complicated, \TR translates the
% view locking into FIB lockings that can be directly implemented on
% relevant switch hardware. Note that the FIBs correspond to the base
% tables from which a user view is derived, therefore, the same SQL
% query that creates the view is used to compile the global view locks
% into the local switch-level locks.

% The rest of the paper is organized as
% follows. Section~\ref{sec:design} outlines the SDN database design,
% featureing \TI and \TR.  Section~\ref{sec:details} describes \TI in
% more details, presents early stage implementation that evaluates the
% \TI design. Section~\ref{sec:discussion} discusses the
% opportunities, challenges, and limits of SDN
% database. Section~\ref{sec:related} connects SDN database to
% relevant OS/PL efforts. Section~\ref{sec:conclusion} concludes the
% paper.

In sum, by leveraging decades of DB research, \Sys introduces the
following features:

\vspace{-1.5mm}
\begin{compactList}
\item \textbf{Customizable and ad-hoc abstractions} that allow
  creation and change of abstractions
  on-demand. (\S~\ref{sec:design},~\ref{sec:details}) \vspace{-1.5mm}
\item \textbf{Realtime verification and synthesis} 
that provide online support for network state verification and ad-hoc
requirement implementation. (\S~\ref{sec:design},~\ref{sec:details})
\item \textbf{Evaluation} that explores scalability of database
  management on data-plane \vspace{-1.5mm}
  % \item \textbf{Transaction processing} that hides concurrency and
  %   recovery
  %   control. (\S~\ref{sec:design})

\end{compactList}