\section{Verification and synthesis}
\label{sec:veri-syn}

This section presents in more details the view interface and their
usage by three examples, namely real-time verification and synthesis
that are fully automatically enabled by database for virtual network,
one big switch, and distributed firewall.

% three example view abstractions, together with the fully automatic
% verification and synthesis services,

\Subsection{Verification and synthesis as data synchronization}

% \todo{(this subsection) HotNet texts, will rework}

A network is in constant change. A virtual view is useful only if its
records are fresh -- reflecting the latest network instantly. For
example, when per-switch rules change, a query on the high-level
policy view \nd{e2e\_routing} (Table~\ref{tb:endpoint}) shall
automatically returns the updated \nd{e2e} reachability. Conversely,
to enable network manipulation via views, the base tables need to be
updated to reflect operations on the views.  For example, to set a new
route \nd{(1,5,4)} for \nd{flow 1} in the example network
(Figure~\ref{fig:eg-one-big-switch} (left)), a user simply insert a
new record denoting the path into the \nd{routing\_policy} view with
\nd{flow} attribute set to \nd{1}. \Sys is responsible of pushing this
abstract view insert into the relevant base \nd{configuration}
inserts.

Generally, view maintenance that keeps virtual views fresh and view
update that synthesizes the base table changes, jointly form a
bi-directional data synchronizer between the base and the view. While
modern DBSes implement view maintenance very efficiently, view update
is supported for restricted
cases~\cite{ak-view-udpate-thesis,relational-lenses}. This is no
surprise, as view update is the harder one: a view contains only
partial information of the base, it is not always possible to locate a
unique base table update~\cite{Bancilhon:view-update-semantics}. To
enable network operations on views, \Sys takes advantage of existing
view maintenance implementation and extends the support of view
updates to network view updates.

\Paragraph{Real-time view maintenance enables network verification.}
\textit{View maintenance} is well supported in modern database systems
(DBS), which \Sys adopts straightforwardly. Specifically, when a view
generated by SQL query program $q$, is queried by a SQL program $p$,
view maintenance translates $p$ on $q$ into queries $p \circ q$ on the
base tables.
% , hence always returning information that is update to the latest
% network state. On the other hand, since views are virtual, This very
% fast re-computation of $p$ keeps the view fresh.  requiring
% re-computation every time it is referred, an alternative is to
% materialize (actually store) the view to accelerate query on the
% views. In this case, view maintenance incrementally updates the
% materialized view table with regard to base table change.
Interestingly, view maintenance offers exactly what is needed in
real-time network verification: by specifying the property of
interests as $p$ over $q$, view maintenance performs check of $p \circ
q$ on network states on the fly. % As shown in \S~\ref{sec:eval},
% checking reachability on a network with more than 10k nodes costs
% less than 10 ms, magnitudes smaller compared to the typically
% per-rule installation or TE operation delay~\cite{b4,ffc}.

% \input{preliminary-eval}
% \subsection{Verification example}
% \subsection{Synthesis example}

\subsection{Virtual network abstraction and enterprise outsourcing}

\begin{sql}
select * from vn_reachability where flow_id = 77899 ;
 flow_id | ingress | egress 
---------+---------+--------
   77899 |     486 |     19  
   ...
\end{sql}

This entry corresponds to a record in configuration as follows:
\begin{sql}
SELECT flow_id, pv FROM configuration_pv WHERE flow_id = 77899;
---------+----------------------------------
 flow_id |                pv                
   77899 | {486,498,462,463,456,472,109,19}
\end{sql}

To update the virtual network policy that revoke transient service for
flow \nd{77899} between ingress \nd{486} and egress \nd{19}, the user
could direcly modify the \nd{vn\_reachability} table by deleting the
corresponding record as follows:
\begin{sql}
DELETE FROM vn_reachability WHERE 
        flow_id = 27079 AND ingress = 486 AND egress = 19;  
\end{sql}

This deletion results in the deletion of three switch-level
configurations as follows:
\begin{sql}
[[462, 463], [486, 498], [498, 462]]  
\end{sql}
The reason that only three entries at switches \nd{462, 486, 498} are
removed is that the rest of the ... are 

\begin{sql}
SELECT flow_id, pv FROM configuration_pv WHERE flow_id = 77899;
 flow_id |                pv                
---------+----------------------------------
   77899 | {483,463,456,472,109,19}
   77899 | {486,498,462,463,456,472,109,19}
   ...
\end{sql}

Similarly, on user request for adding new or updating exiting virtual
network end to end policy, \Sys synthesizes the relevant switch-level
configuration modification.
Add new policy:
\begin{sql}
INSERT INTO vn_reachability VALUES (55716, 557, 483);  
\end{sql}

Update policy:
\begin{sql}
UPDATE vn_reachability SET egress = 230
        WHERE flow_id = 97940 AND ingress = 497;  
\end{sql}

Finally, it is worth noting that, a policy update request may not
always be valid. For example, a deletion of policy \nd{(42692, 497,
  375)} in ... 

\begin{sql}
SELECT source, target, pv FROM configuration_pv WHERE flow_id = 42692;
 source | target |                pv                 
--------+--------+-----------------------------------
    486 |    375 | {486,498,462,463,456,472,108,375}
    497 |    230 | {497,462,463,456,230}
    497 |    375 | {497,462,463,456,472,108,375}
    ...
\end{sql}

While \Sys, which utilizes the default view updates mechanism
implemented by postgres, will simply remove policy \nd{(42692, 497,
  375)} from \nd{vn\_reachability} view, and leaves the per-switch
configuration unchanged. 
% \begin{sql}
% Switch delta after del of 42692 between 497 and 375 []
% \end{sql}

\subsection{One big switch abstraction and distributed firewall}

Set default switch 
\begin{sql}
ALTER VIEW reachability_rel_obs_out2 ALTER COLUMN source SET DEFAULT 591;  
\end{sql}


Dynamically pick switch that is closest to destination
\begin{sql}
INSERT INTO reachability_rel_obs_out2  (flow_id, source, target)
SELECT flow_id, source, target FROM
       (SELECT * FROM
       	       (SELECT 89406 as flow_id,
	       	       switch_id as source,
	       	       483 as target, 
		       (SELECT count(*) FROM
		        pgr_dijkstra('SELECT 1 as id, switch_id as source,
					     next_id as target,
					     1.0::float8 as cost
			              FROM topology', switch_id, 483,FALSE, FALSE)
                       ) AS hops 
                FROM obs_nodes
	       ) AS tmp WHERE hops !=0
       ) AS tmp2 ORDER by hops LIMIT 1;  
\end{sql}
