\section{View Interface}
\label{sec:details}

The design goal of \Sys view interface is to combine the strength of
the following:
\begin{itemize}
\item An interface that loads and transforms the dataplane states.
  % structured in a format that eases control logic.
  The challenge is to
  identify a small set of views that are also expressive enough for
  common networking tasks. (\S~\ref{subsec:view-library})
\item Composition. (\S~\ref{subsec:compose})
\item Built-in services of real-time verification and
  synthesis. (\S~\ref{sec:veri-syn})
\item Performance. The challenge is that a naive implementation of
  relational queries does not scales well for the networking setting
  of SDN, where most interesting abstractions, by nature, will call
  for path-related computation that is recursive, and hence expensive
  in the naive implementation. (\S~\ref{sec:eval})
\end{itemize}

To achieve all the above, \Sys features a library of view primitives.
... create abstractions on the fly that can be categorized in two
groups: (1) the per-flow views ... We also call this type ``local
views'', because ... (2) the network-wide views ... we call this type
``global views''.  (1) enables real-time verification and synthesis;
(2) provides the interface for integrating network-wide service such
as traffic engineering.  This section presents (1,2) in
details. Services and performance are discussed in
\S~\ref{sec:veri-syn} and \S~\ref{sec:eval}.

\subsection{Abstraction hierarchy}

\label{subsec:view-library}

\subsection{Base tables}

Before introducing the derived views, let us first discuss in depth
the base tables -- the flat universe of networking data that are
actually stored in the database. The base tables serve two roles: (1)
a flat universe of tables to which all network configuration states
and state changes can be easily loaded; (2) the bases for building
layers of view abstractions where the reverse view update shall be
(relatively) easy.

\begin{sql}
CREATE OR REPLACE FUNCTION reachability_perflow(f integer)
RETURNS TABLE (flow_id int, source int, target int, hops bigint) AS 
$$
BEGIN
	DROP TABLE IF EXISTS tmpone;
	CREATE TABLE tmpone AS (
	SELECT * FROM configuration c WHERE c.flow_id = f
	) ;

	RETURN query 
        WITH ingress_egress AS (
		SELECT DISTINCT f1.switch_id as source, f2.next_id as target
       	      	FROM tmpone f1, tmpone f2
	      	WHERE f1.switch_id != f2.next_id AND
		      f1.switch_id NOT IN (SELECT DISTINCT next_id FROM tmpone) AND
	              f2.next_id NOT IN (SELECT DISTINCT switch_id FROM tmpone)
                ORDER by source, target),
	     reach_can AS(
                SELECT i.source, i.target,
	      	       (SELECT count(*)
                        FROM pgr_dijkstra('SELECT 1 as id,
			     	           switch_id as source,
					   next_id as target,
					   1.0::float8 as cost FROM tmpone',
			     i.source, i.target,TRUE, FALSE)) as hops
	        FROM ingress_egress i)
	SELECT f as flow_id, r.source, r.target, r.hops FROM reach_can r where r.hops != 0;
END
$$ LANGUAGE plpgsql;
\end{sql}

\subsection{Virtual views}

\subsubsection{View primitives}

\Paragraph{Per-flow forwarding graph and reachability views}

Lies in the heart of any pair-wise reachability views, be it
reachability for a plain network, for an one-big-switch network, or
for an arbitrary virtual network, is a query like the following:
 
% \begin{sql}
% SELECT source, 
%        target,
%        (SELECT count(*)
%         FROM pgr_dijkstra('SELECT * FROM """ + fg_view_name + """',
%              source, target, TRUE, FALSE)) AS hops
% FROM ingress_egress_pairs
% \end{sql}

% \begin{sql}
% def generate_forwarding_graph (cursor, flow_id):

%     fg_view_name = "fg_" + str (flow_id)

%     try:
%         cursor.execute("""
%         CREATE OR REPLACE view """ + fg_view_name + """ AS (
%         SELECT 1 as id,
%                switch_id as source,
% 	       next_id as target,
% 	       1.0::float8 as cost
%         FROM configuration
%         WHERE flow_id = \%s
%         );
%         """, ([flow_id]))

%     except psycopg2.DatabaseError, e:
%         print "Unable to create fg_view table for flow " + str (flow_id)
%         print 'Error \%s' \% e    
% \end{sql}

% \begin{sql}
% def generate_reachability_perflow (cursor, flow_id):

%     fg_view_name = "fg_" + str (flow_id)
%     reach_view_name = "reachability_" + str (flow_id)

%     try:
%         cursor.execute("""
%         CREATE OR REPLACE view """ + reach_view_name + """ AS (
%           WITH ingress_egress AS (
%                SELECT DISTINCT f1.source, f2.target
%                FROM """ + fg_view_name + """ f1, """ + fg_view_name + """ f2
%                WHERE f1.source != f2.target AND
%                      f1.source NOT IN (SELECT DISTINCT target FROM """ + fg_view_name +""") AND
%                      f2.target NOT IN (SELECT DISTINCT source FROM """ + fg_view_name +""" )
%                ORDER by f1.source, f2.target),
%                ingress_egress_reachability AS (
%                SELECT source, target,
%                       (SELECT count(*)
%                        FROM pgr_dijkstra('SELECT * FROM """ + fg_view_name + """',
%                                          source, target, TRUE, FALSE)) AS hops
%                FROM ingress_egress)
%           SELECT * FROM ingress_egress_reachability WHERE hops != 0
%         );""")

%     except psycopg2.DatabaseError, e:
%         print "Unable to create reachability table for flow " + str (flow_id)
%         print 'Error \%s' \% e      
% \end{sql}

% This section presents a prototype design of \TI in more details,
% showing preliminary evaluation with promising results.

% \Paragraph{Update views}



\subsubsection{View composition}
\label{subsec:compose}

% \Paragraph{One big switch}

From user perspective, a network view is a derived table that offers
the same tabular interface as ordinary (\ie materialized table that is
actually stored) tables. This allows to a stacking of views ...  ...
We use ``composition'' to loosely refer to the derivation of views
from existing views. ...

Consider one big switch, its per-flow forwarding graph view can be
built on top of per-flow forwarding graph view and one big switch
topology view, as follows:
\begin{sql}
CREATE OR REPLACE VIEW obs_1_fg_36093_3 AS (
       select 1 as id,
       	      switch_id as source,
	      next_id as target,
	      1.0::float8 as cost
       FROM obs_1_topo, fg_36093
       WHERE switch_id = source AND next_id = target
       ORDER BY source, target
);
\end{sql}

Note that, \nd{fg\_36093} is used as a filter for selecting
\nd{obs\_1\_topo} records that ... Symmetrically, one could also
``reverse'' the filtering and use \nd{fg\_36093} as a filter instead,
as follows:
\begin{sql}
CREATE OR REPLACE VIEW obs_1_fg_36093_4 AS (
       select 1 as id, source, target, 
	      1.0::float8 as cost
       FROM obs_1_topo, fg_36093
       WHERE switch_id = source AND next_id = target
       ORDER BY source, target
);
\end{sql}


While the above two view composition, namely SQL join, is very
intuitive: its body of ``view join'' directly translates. It is no
longer updatable. ... As an alternative equivalent view is by
... where one table is used as an ... parameter ...
\begin{sql}
CREATE OR REPLACE VIEW obs_1_fg_36093_2 AS (
       select 1 as id,
       	      switch_id as source,
	      next_id as target,
	      1.0::float8 as cost
       FROM configuration NATURAL JOIN topology
       WHERE flow_id = 36093 AND subnet_id = 1
       ORDER BY source, target
);
\end{sql}

To make advantage of view updates, \Sys adopts this
approach. Unfortunately, views are not allowed to be paramterized in
SQL standard. To achieve the affect of parameterized views, \Sys uses
Python wrapper as walk-round, as follows:

