\Section{Examples}

\noindent{}This section discusses \Sys features by examples. 

% including customizability, realtime verification and synthesis
% support, transaction processing, and federated cooperation.

\Paragraph{Enterprise outsourcing:} Enterprise network today demands
new functionalities beyond the traditional end to end connectivity
such as security and privacy.
% , requiring expertise beyond the enterprise's scope.
Enterprises are also moving their networks to cloud, leasing virtual
networks from (multiple) remote datacenters rather than hosing a local
infrastructure, which further complicate the issue. In response,
enterprise outsourcing is emerging, which separates who own the
network and who manage it by outsourcing the management task to a
third party
expert~\cite{middlebox-outsourcing,platform-service-model}. Existing
proposals, \eg
one-big-switch~\cite{platform-service-model,optimize-obs} and network
virtualization, expose enterprise network through unified primitive
APIs. % There also exist tools that map external
% solutions back to enterprise network.
A network abstraction with easy, flexible, and full control is still
missing. Can the enterprise expose only network aggregates, \eg
average bandwidth, or even just a range, without revealing the
sensitive details?  Can the enterprise adopt application-specific
abstractions, and change abstractions as business goes?

To this end, \Sys abstractions is \textit{\textbf{customizable}},
easily created and destroyed by the user: \Sys abstraction does not
enforce commitment to pre-defined freezing abstraction, granting users
full control over what and how an outside party sees his network. By
using SQL as abstraction definition and manipulation language,
\Sys also requires lesser computer specialty, lowering the bar for
adoption.

% The view and view update mechanism of SDN database smoothly fit both
% roles. Specially, by keeping the network as a database, rather than
% exposing every detail of the network, the owner could create a view,
% \eg a one-big-switch view that hides internal sensitive information,
% and exposes the view instated, upon which a solution is composed by
% the external expert. SDN database then uses view update mechanism to
% compile the expert solution into the actual operations on the
% owner's network.

\Paragraph{Datacenter provisioning:} In a datacenter that provides
elastic cloud computing service~\cite{aws-ec2,vm-sla,Portland}, leases
virtual network that is either pre-configured or customizable, the
administrator is facing two problems: network provisioning, \eg
improving utilization by migrating VMs while respecting resource
limits and customer SLAs; and service implementation, \eg compiling
customer's logical network configuration into the actual datacenter
infrastructure.  Virtualization
techniques~\cite{nv-survey,scalable-nv-sdn,virtual-forwarding-plane}
automate many primitive operations that ease datacenter
tasks. However, existing primitives are too primitive to achieve
automatic online datacenter management. Can the administrator inspect
the network state, checking high-level network-wide properties? Can
the operator implement a customer's ``non-standard'' request without
composing a complete program out of primitive wrapping APIs?

% \Sys can automate the above tasks by user-defined data abstraction,
% capturing the customer's logical network and network-wide properties
% being provisioned in a unified way.

To this end, \Sys performs \textit{\textbf{realtime network
    verification and synthesis}} directly on user-defined
abstractions. By the bi-directional data synchronizer between
abstraction and the data-plane, administrator analyzes network state
(\eg check SLA conformance) by issuing a query over \Sys abstraction,
as simple as a SQL select statement. Conversely, arbitrary customer
operations on logical network is automatically populated into to the
datacenter without the administrator's
supervision. % Preliminary evaluation results (\S~\ref{sec:eval})
% show that the typical DB-induced latency per switch rule update on a
% VL2 network of up to 10'000 switches is at the scale of 1-10 ms.

% or external state-exploration tools that is usually sensitive to
% network seize \TI automates these tasks as follows: load the
% datacenter network configuration (flow tables, topology \etc) into
% base tables; develop various provisioning/service \textit{views}
% that exposes resource utilization and SLAs status (violation) -- the
% views are either selected by administrator from a pre-defined pool,
% or introduced on the spot as a SQL query; and finally develop
% \textit{triggers} that implements our generic view update algorithm.
% Once the SDN database is set up, view maintenance allows the
% datacenter administrators to verify SLA conformance and resource
% limit enforcement in realtime; whereas view update adjusts the
% datacenter configurations according to changes to SLA or customers'
% requirements expressed over views.

\Paragraph{Distributed firewalls:} Firewall offers a direct
abstraction for specifying and enforcing network policy beyond
standard routing. However, conventional firewall functionality relies
on topology constraints, \eg assuming that devices on the protected
side are trusted.  The placement of the firewall also introduces a
choke-point to network performance such as throughput.  Distributed
firewalls is
proposed~\cite{impl-distributed-firewall,distributed-firewall} to
mitigate these shortcomings, implemented by a high-level language that
specifies centralized firewall policy, and a mechanism that
distributes the policy into end nodes.  Such implementations, no
matter how carefully designed to assure functional correctness, leaves
behind concurrency and recovery control. While one may leverage
existing consistent network update solutions for concurrency control,
which, nevertheless, incurs overhead on network device or is too slow
for online use.  Recovery control that correctly roll back the network
in case of failure is also missing.

To free users from the challenging concurrency and recovery control,
\Sys supports \textit{\textbf{transaction processing}}, where a
centralized firewall operation is processed as a single logic
transaction that is distributed over the network, by adapting
transaction processing from distributed DB research, such as two-phase
locking to achieve ACID semantics.
% A novel aspect is that, to active reasonable system performance,
% rather than adding control and recovery control to the abstract
% entity over which policy is specified, \Sys uses the SQL query that
% generates the abstraction to translate the transaction into
% ``concrete ones'' processed at the individual end nodes, where
% transaction processing is achieved autonomously. Hence, \Sys assumes
% and advocates that network devices, such as switches, shall be built
% with basic transaction mechanism (\eg locking).


% \Sys uses DB techniques to achieve transaction processing of the
% right tool, view abstraction allows user define requirement on
% arbitrary high-level entities, and view update compiles view
% policies into the constituting network devices. Finally, \TR
% enforces that the distributed policies are correctly applied to
% packets.
% transactional processing

\Paragraph{Integrable networking:}
\label{ssec:integration-integration}
Previously we demonstrate \Sys features that separate high-level
centralized user operation and low-level distributed implementation in
a unified manner in a vertical
architecture.% , as well as cater for specific needs such as
% customizability, realtime support, and concurrency / recovery, in a
% vertical network architecture
This example shows that \Sys also enables network integration
horizontally.  Consider merging two existing enterprise sites when two
company department merges, or a network of independently administrated
networks in the case of SDX (Software-defined Internet
Exchange)~\cite{sdx}, or the inter-operation of multiple controllers
in a SDN-powered virtualized networks. To achieve a consistent
inter-site network behavior, resolve potential conflicts, while
minimizing the modification imposed at, maximizing the autonomy
demanded by individual participant, what is the desirable
``collaboration interface''? 

These problems are challenging but not new, seen by DB community in
1990s when data integration research emerged to connect autonomous
DBSes. We believe that, by lending itself to the rich
\textbf{\textit{data integration}} research such as federated
DB~\cite{Sheth:1990:federated-DB}, semi-structured
DB~\cite{lore,semistructured}, \Sys brings hope to a natural
solution. Intuitively, the inter-site/controller network is managed
through a super-abstraction, constructed from the per-site
abstractions, in a way no different from the per-site ones. The
super-abstractions form a public interface, where the per-site
abstractions are for private management, hiding participant's internal
data and isolating each another. Features of customizability,
real-time support, and transaction processing also extend to the
super-abstraction.
